%!TEX root = ../dissertation.tex
\chapter{Conclusion}
\label{conclusion}

Throughout the present dissertation we have introduced a novel model that aims to solve and model one of the reference problems in the quest for achieving a complete comprehension of the relation between Computer Vision and Natural Language Processing. While the DMN is based on previous model ideas, the specific novel contributions, such as the inclusion of dynamic filters, the implementation of recurrent units based on SRU instead of LSTM ones, the inclusion of all language and vision features to represent multimodal features and the incremental upsampling module with mirror links incoming from the visual module do improve the overall performance on the task, as it has been demostrated on three of four datasets. While the model has a state-of-the-art performance, it has different drawbacks and limitations, which are directly related to the language modelling process as the experiments have shown. It is expected that if better language features are trained and exploited, the model should improve its performance.

We have also proposed an scalable and generic backend architecture to handle and process Computer Vision applications from a client-server perspective. As the architecture is scalable, it should handle multiple concurrent requests, and as it accepts any PyTorch model, it can be used to deploy large-scale CV applications in production. As a result, we have shown the performance of the DMN architecture on the real world, by means of the DMNApp, an Android application that enable users to take pictures from their devices/camera and process them using custom queries, while visualizing their results on the actual device. While the proof-of-concept application can be used for research and visualization purposes initially, it can be extended easily to handle real tasks, such as image classification and clustering based on natural language expressions, or object reconstruction based on language using multiple images.

As the object segmentation based on natural language task, as member of the large family of Vision \& Language problems is far from being a closed problem, we expect that the research developments done and presented on this document also help to provide some insights to other recognition problems, such as scene generation from natural language and visual object relationship detection. As the frontier of visual recognition extends to cover other problems, it will be possible to take novel ideas and methods to improve performance on this task on further works on the future.
